\documentclass[UTF8]{ctexbeamer}
\usetheme{Madrid}
%\usecolortheme{beaver}
\usepackage{graphicx}
\usepackage{multicol}
\usepackage{subfigure}

% \newcommand{\bf}[1]{\textbf{\emph{#1}}}
% \newcommand{\img}[2]
% {
%     \begin{figure}
%         \centering
%         \includegraphics{\emph{#1}}
%         \caption{\emph{#2}}
%     \end{figure}
% }

\DeclareGraphicsExtensions{.eps,.ps,.jpg,.bmp,.png}
%%Information to be included in the title page:
%\title{X}
%\author{Deli}
%%\institute{Deli institute}
%\date{\today}
\title[眼睛] %optional
{眼睛}
% [] {} 两个还是有区别的,具体的差异尝试编译一下就了解了。

\subtitle{第九章}
% 副标题,beamer的默认模板中基本上这些内容都有

\author[]{徐骄阳} % (optional, for multiple authors)
% {Deli1\inst{1} \and Deli2\inst{2}}

\institute[系统科学学院]{系统科学学院} % (optional)
% {
% 	\inst{1}%
% 	Faculty of Physics\\
% 	Very Famous University
% 	\and
% 	\inst{2}%
% 	Faculty of Chemistry\\
% 	Very Famous University
% }

\date[\today]{} % (optional)
%在每个section 前边单独插入当前章节高亮的目录页(当然最原始的目录页你还是需要手动录入的,不要想偷懒)
\AtBeginSection[]
{   
    \begin{frame}
        \frametitle{目录}
        \begin{multicols}{2}    
            \tableofcontents[currentsection,hideallsubsections]
        \end{multicols}
    \end{frame}

	% \begin{frame}
	% 	\frametitle{梗概}
	% 	\tableofcontents[hideallsubsections]
	% \end{frame}
}
\begin{document}
\frame{\titlepage}
\section*{引言}
\begin{frame}
    \frametitle{引言}
    \begin{itemize}
        \item  \textbf{视觉}(vision)对于动物而言非常重要,因为它赋予了生命感知周围物体的一个途径。
        通过视觉一个智能体便能接收光信号进行处理后,对环境的刺激做出反应。视觉的显著意义也许可以从人体大脑皮层约有一半区域介入对视觉世界的分析
        事实得到最佳的证明.
        \item 哺乳动物的视觉系统始于\textbf{眼睛}。 眼睛的背部是\textbf{视网膜} (retina) ,
        这里拥有特化的、将光能转化成神经活动的光感受器 。眼睛的其余部分像照相机,
        在视网膜上形成清晰的、但是并不永久的图像。眼睛可以自动调节亮度的差异,
        并自动聚焦于所感兴趣的物体。眼睛还具有诸如通过眼动对移动目标的跟踪,以及通过过眼泪和眨泪使其透明表面保持清洁的能力。
    \end{itemize}
\end{frame}

\begin{frame}
    \frametitle{本章主要内容}
    在本章中,我们将对眼睛和视网膜进行考察:
    \begin{itemize}
        \item 光怎样将信息携带至我们的视觉系统;
        \item 眼睛怎样将图像形成于视网膜上;
        \item 视网膜怎样将光能转化为神经信号以提取关于亮度和颜色差异的信号
    \end{itemize}
\end{frame}
\section{光的特性}
\subsection{光}
\subsection{光学}
\section{眼睛的结构}
\subsection{大体解剖结构}
\subsubsection{眼底镜特性}
\subsection{横切面解剖结构}
\section{眼睛中图像的形成}
\subsection{角膜的折射}
\begin{frame}
    \frametitle{眼睛中图像的形成}

    眼睛接收环境中由物体发射或反射的光线,并将他们聚焦于视网膜以成像,如图\ref{pic3-1}。光在穿过角膜弯曲的表面时候
    会发生偏折。自折射表面至平行光的汇聚点的距离称为\textbf{焦距}焦距和角膜的曲率有关:弯曲程度越高,焦距越短。
    焦距的倒数称为\textbf{屈光度}(diopter)。角膜的折射力约42屈光度。
    \begin{figure}
        \centering
        \includegraphics[height=0.3\textwidth]{img/pic3-1.png}
        \caption{角膜的折射作用\label{pic3-1}}
        % [width=0.8\textwidth,height=0.5\textwidth]
    \end{figure}

\end{frame}

\subsection{晶状体的适应性调节}
\subsection{瞳孔对光反射}

\begin{frame}
    \frametitle{晶状体与瞳孔的调节作用}
    \begin{itemize}
        \item 晶状体参与距离眼睛9m之内物体的清晰成像,当物体逐渐接近时,通过调节晶状体的屈光率便可将光线聚焦在视网膜上,如图\ref{pic3-2}。
        \item 瞳孔的收缩增加了聚焦深度,有助于远处物体的清晰成像,搜索瞳孔使远处的模糊影像更接近一个点,这样物体的聚焦情况会得到改善。
    \end{itemize}
    \begin{figure}
        \centering
        \includegraphics[height=0.3\textwidth]{img/pic3-2.png}
        \caption{适应性调节\label{pic3-2}}
    \end{figure}
\end{frame}

\subsection{视野}
\subsection{视锐度}

\begin{frame}
    \frametitle{视野}
    \begin{itemize}
        \item 眼睛的结构以及它在头部的位置使得特定时间内看到的外部世界收到限制。如图\ref{pic3-3},\textbf{视野}是指眼睛直视前方时视网膜所能看到的全部空间范围。值得注意的是,视野中的物体在视网膜上是反转的。
        \item 眼睛辨别两个相邻点的能力称为\textbf{视锐度}(visual acuity),主要取决于光感受器在视网膜上的分布和眼睛的折射精度。
        \item 视网膜上的距离可以用\textbf{视角}(visual angle)的度数来表示。当辨认视角为0.083°的字母时,视力为20/20
    \end{itemize}

    \begin{figure}
        \centering
        \subfigure[单眼视野\label{pic3-3}]{\includegraphics[height=0.25\textwidth]{img/pic3-3.png}}
        \subfigure[视角\label{pic3-4}]{ \includegraphics[height=0.25\textwidth]{img/pic3-4.png}}
    \end{figure}
\end{frame}

\section{视网膜的显微解剖}
\subsection{视网膜的分层组构}
\subsection{光感受器的结构}
\subsection{视网膜结构的区域差异}

\section{光传导}
\subsection{视杆中的光传导}
\subsection{视锥中的光传导}
\subsection{暗适应和明适应}
\section{视网膜的信息处理}
\section{视网膜的输出}

\subsection{神经细胞感受野}
\begin{frame}
    \frametitle{神经节细胞的感受野}
    % \begin{columns}
        % \column{.5\textwidth}{
            中心-周边感受野的组织方式从双极细胞通过内网状层的突触传向神经节细胞。
            内网状层的无长突细胞的侧向链接也参与神经节细胞感受野的形成,但至今我们对这些连接的具体作用知之甚少。
        
            多数神经节细胞具有和上述双极细胞一样的同心圆式的中心-周边感受野结构。给光中心和撤光中心神经节细胞接收同类双极细胞的输入。
        
            多数视网膜神经节细胞对同时覆盖其感受野中心和周边的光刺激变化并无反应,主要对他们感受野内的\textbf{亮度差异}有反应。        
        % }\column{.5\textwidth}{
            \begin{figure}
                \centering
                \includegraphics[width=.8\textwidth]{img/pic7-1.png}
                \caption{神经节细胞的中心-周边感受野\label{pic7-1}}
            \end{figure}
            \tiny{
                当一个暗点投射在撤光中心型神经节细胞的感受野中心时,细胞发放一串动作电位,但暗点范围扩大,则放电大幅减少。
            }
        % }
    % \end{columns}
    
\end{frame}

\begin{frame}
    \frametitle{神经节细胞感受野对明暗边界的调制}
    考察明暗边界对撤光中心神经节细胞输出的影响。根据实验,该信号有如图\ref{pic7-2}所示反应。%#TODO:做个图像分析
    \begin{figure}
        \centering
        \includegraphics[width=.8\textwidth]{img/pic7-2.png}
        \caption{神经节细胞对感受野内明暗边界的反应\label{pic7-2}}
    \end{figure}
    从神经节细胞感受野的组织形式,我们可以推断,世界系统特化为对局部空间变化进行检测,而不是对于落在视网膜上光的绝对幅度进行检测,因此,\textbf{视网膜对光或暗的感知是相对的}。
\end{frame}
\subsection{神经细胞的类型}
在哺乳类视网膜,多数数神经节细胞具有一个中心-周边感受野,其中心
区域或具给光反应或具撒光反应。他们可以进一步根据外形、突触连接、和
电生理特性进行分类。

譬如,基于胞体大小与树突结构
\subsection{并行处理}

\section*{结语}
\begin{frame}
    \frametitle{小结}
    \begin{itemize}
        \item 本章的第一部分主要讨论了角膜、瞳孔、晶状体眼睛对外界光辐射的调节作用,使得光线恰好聚焦至视网膜上。
        \item 在第二部分我们主要讨论了视网膜的光感受器结构如何将光信号接收并转化为神经信号以及眼睛的明暗适应与色觉产生机制。
        \item 在第三部分我们讨论了视网膜的细胞组成以及信息传递通路,详细考察了视网膜的中心-周边感受野表征机制。
        \item 在第四部分我们简单概述了哺乳动物的神经节细胞类型以及视网膜中的并行处理。
    \end{itemize}
\end{frame}
\end{document}